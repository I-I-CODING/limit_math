\documentclass[UTF8, a4paper, 12pt]{ctexart}
\setCJKmainfont{Noto Serif CJK SC}
\usepackage{amsmath}
\usepackage{amssymb}
\usepackage{amsthm}
\usepackage{mathtools}
\setlength{\parindent}{0pt}
\pagestyle{plain}

\begin{document}

\title{高阶导数}
\date{}
\maketitle

\section{高阶导数的定义}

\[f''(x) = \lim_{\Delta x \to 0}\frac{f'(x + \Delta x) - f'(x)}{\Delta x}\]

可记为:
\[f''(x), y'', \frac{d^2y}{dx^2}, \frac{d^2f(x)}{dx^2}\]

n阶导数的拉格朗日记法:
\[f^{n}(x), y^{n}(x)\]

\section{高阶导数的计算}
\subsection{直接法}

eg.1
\[y^{(n)} = (e^{\lambda x})^{(n)} = \lambda^n e^{\lambda x} \quad n = 0, 1, 2, \dots \]

eg.2
\[\sin(\omega x) ^ n = \omega ^ n \sin(\omega x + n\frac{\pi}{2}) \quad n = 0, 1, 2, \dots\]
\[\sin(x)^n = \sin(x + n\frac{\pi}{2}) \quad n = 0, 1, 2, \dots \]
\[\cos(x)^n = \cos(x + n\frac{\pi}{2}) \quad n = 0, 1, 2, \dots \]
\[\cos(\omega x) ^ n = \omega ^ n \cos(\omega x + n\frac{\pi}{2}) \quad n = 0, 1, 2, \dots\]
eg.3

\subsubsection{第一组结论:}
\[f(x) = a_0x^n + a_1x^{n - 1} + \dots + a_{n -1}x + a_n\]
\[f^{n}(x) = a_0n! \qquad f^{k}(x) = 0, (k > n)\]

\subsubsection{第二组结论:}
\[
(x^m)^{(n)} =
\begin{cases}
m(m - 1)\dots(m - n + 1)x^{m - n} & n < m \\
n! & n = m \\
0 & n > m
\end{cases}
\]

\subsubsection{第三组结论}

对于:
\[y = (x+c)^{\mu}\]

有:
\[y^{(n)} = \mu(\mu - 1)(\mu - 2)\dots(\mu - n + 1)(x + c)^{\mu - n} \quad (n = 1, 2, \dots)\]

当$\mu = 1$时:
\[(\frac{1}{x+c})^{(n)} = \frac{(-1)^nn!}{(x+c)^{(n+1)}}\]

对于对数:
\[[ln(1+x)]^{(n)} = (\frac{1}{1+x})^{(n-1)} = \frac{(-1)^{n-1}(n-1)!}{(1+x)^n}\]

eg.4

设$y = f(\ln x)^{2}$, 求$y''$.

解:

$y'  = 2f(\ln x) \cdot f'(\ln x) \cdot \frac{1}{x}$

$y'' = [2f(\ln x)]' \cdot f'(\ln x) \cdot \frac{1}{x} + 2f(\ln x) \cdot [f'(\ln x)]' \cdot \frac{1}{x} + 2f(\ln x) \cdot f'(\ln x) \cdot [\frac{1}{x}]'$

\subsection{补充公式:}
\[(a^{x})^{(n)} = a^{x}(\ln a)^{n} \quad (a > 0)\]
\[(\frac{1}{ax + b})^{(n)} = \frac{(-1)^{n}n!a^{n}}{(ax + b)^{n + 1}} \quad (a \neq 0)\]

\subsection{间接法}
设$\mu = \mu (x) , \nu = \nu (x)$ 具有n阶导数, 则

\begin{align}
(\mu \pm \nu)^{(n)} &= \mu^{(n)} \pm \nu^{(n)} \\
(c\mu)^{(n)} &= c\mu^{(n)} \quad (c\in\mathbb{R}) \\
(\mu \nu)^{(n)} &= \sum_{k = 0}^n \binom{n}{k} \mu^{(n - k)} \nu^{(k)} \\
(\mu + \nu)^n &= \sum_{k=0}^n \binom{n}{k}\,\mu^{\,n-k}\nu^{\,k}
\end{align}


eg.1

求$y = \frac{x}{x^2 - 1}$的n阶导数
\begin{align*}
y &= \frac{x}{(x+1)(x-1)} \\
  &= \frac{1}{2}\left(\frac{1}{x+1} + \frac{1}{x-1}\right) \\
\therefore\quad y^{(n)} &= \frac{1}{2}\left((\tfrac{1}{x+1})^{(n)} + (\tfrac{1}{x-1})^{(n)}\right) \\
  &= \frac{1}{2}\left(\frac{(-1)^n n!}{(x+1)^{(n+1)}} + \frac{(-1)^n n!}{(x-1)^{(n+1)}}\right)
\end{align*}

eg.2

求$y = (\sin x)^6 + (\cos x)^6$的n阶导数
\begin{align*}
y &= (\sin^2 x)^3 + (\cos^2 x)^3 \\
    &= (\sin^2 x + \cos^2 x)(\sin^4 x - \sin^2 x \cos^2 x + \cos^4 x) \quad \text{(立方和公式)}\\
    &= (\sin^2 x + \cos^2 x)^2 - 3\sin^2 x \cos^2 x   \quad \text{(展开)}\\
    &= 1- \tfrac{3}{4}\sin^2 2x    \quad \text{(二倍角公式)}\\
  &= \frac{5}{8} + \frac{3}{8}\cos 4x   \quad \text{(后续带公式)}
\end{align*}

eg.3

已知$y = \arctan x$, 求$y^{(n)}(0), n > 1$

\begin{align*}
y' = \frac{1}{1 + x^2} \\
\intertext{整理得:}
(1 + x^2)y' = 1 \\
\intertext{对等式两边同时求n阶导,由莱布尼兹公式:}
\sum_{k = 0}^n \binom{n}{k} (1 + x^2 )^{(k)}y^{(n -k + 1)} = 0 \\
\intertext{仅当 $k=0,1,2$ 时,$(1 + x^2)^{(k)}\neq 0$}
\binom{n}{0} (1 + x^2)y^{(n+1)} + \binom{n}{1}(2x)y^{(n)} + \binom{n}{2}(2)y^{(n - 1)} = 0 \\
\intertext{简化得: } \\
(1+x^2)y^{(n+1)} + 2nxy^{(n)} + n(n - 1)y^{(n - 1)} = 0 \\
\intertext{令$x = 0$, 代入得:} \\
y^{(n+1)}(0) = -n(n - 1)y^{(n -1)}(0) \\
\intertext{令$a_m = y^{(m)}(0)$, 则递推关系为:} \\
a_m = -(m - 1)(m - 2)a_{m - 2}\quad m\geq 2 \\
a_0 = y(0) = \arctan(0) = 0,a_1 = y'(0) = 1 \\
\intertext{当m为偶数时,} \\
a_m = 0; \\
\intertext{当m为奇数时, } \\
a_m = (-1)^{\frac{m-1}{2}}(m -1)! \\
\intertext{当 $n > 1$ 时:}
y^{(n)}(0) =
\begin{cases}
0, & \text{若 $n$ 为偶数},\\
(-1)^{\frac{n-1}{2}}(n - 1)!, & \text{若 $n$ 为奇数}.
\end{cases}
\end{align*}

\subsubsection{莱布尼兹公式的使用建议}
\begin{enumerate}
  \item 求两个函数乘积的高阶导数
  \item 建议一个是幂函数或多项式函数
\end{enumerate}

\end{document}